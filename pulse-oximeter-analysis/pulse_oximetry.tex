\clearpage
\section{Pulse Oximetry}\label{section:pulse_oximetry}

Pulse oximetry is a non-invasive optical method used for continuous measurement of oxygen saturation of arterial blood. It has become a monitoring standard for patients in critical care and anesthesia during the past two decades because of its capability to measure the oxygen saturation of arterial blood continuously and non-invasively, providing vital information about the cardiorespiratory function of the patient and thus making it a convenient, safe and effective source of alarms. In many applications it is the fastest and the most efficient way to detect hypoxia and other life-threatening conditions that require a fast response.\ \cite{Webster1997}

The method is rather accurate: when the arterial oxygenation is over 70\%, pulse oximeters usually provide a measurement that differentiates from the actual oxygenation by less than 3\%-units. This level of accuracy is enough for most applications as the usual operating range is well over 80\% level.\ \cite{Bowes1989}

The history of oximetry dates back roughly 70 years. The first motive for its development was to monitor fighter pilots when flying at high altitudes during World War II - at high speeds and g-forces their very life depended on them staying conscious. The early devices used mercury lamps, different filters and a spectrophotometer and were not that accurate over time, but they served their purpose even though they were unable to differentiate between arterial and venous blood.\ \cite{oximetry.org}

The era of modern pulse oximetry saw light in the early 1970's when it was discovered that the pulsatile nature of the oximeter output could be used to extract arterial oxygen saturation from the signal. Soon after that, an American corporation called Biox made another breakthrough by introducing LEDs as the light source in oximeters; incidentally, Biox is now a part of GE through several acquisitions. The technology has evolved tremendously since the early days but the principle has remained the same since the 1980's. In addition to reducing size and cost, the method has been extended by adding wavelengths to measure total hemoglobin concentration and other blood parameters as well as improving the accuracy of basic S\subscript{p}O\subscript{2}, but the standard two-wavelength solution for measuring the percentage of arterial oxyhemoglobin is still by far the most commonly used one.\ \cite{Aoyagi2007}%Aoyagi puhuu multiwavelengthist�

\subsection{Theory of Pulse Oximetry}

An oximeter is physically little more than a light absorption meter. Placed on two sides of a tissue bed, it emits monochromatic light from one side and measures the light intensity on the other side. The operation is repeated with another wavelength. The absorption rates of the two wavelengths are different due to the uneven absorption spectra of different types of hemoglobin (figure \ref{fig:absorption_curves}) and the ratio of the absorptions can be used directly to estimate blood hemoglobin oxygenation. Unfortunately the resulting figure represents the sum of arterial and venous blood and the tissue bed itself and without carefully calibrating the device for each patient doesn't tell the observer much about the patient's condition. A signal processing method must be used to extract the arterial blood hemoglobin oxygenation from the total value.

The wavelength range over which this method can be used is limited between 600 and 1300 nm. At wavelengths shorter than 600 nm a large portion of the light is absorbed into the melanin of the skin, whereas wavelengths over 1300 nm are absorbed by water in tissues. The most commonly used wavelengths in two-wavelength systems are 660 nm and 940 nm - they are roughly the local absorption minimum and maximum, respectively, of oxyhemoglobin, and on the other hand deoxyhemoglobin's absorption spectrum is a descending one on that range (figure \ref{fig:absorption_curves}). Therefore these two wavelengths offer the best resolution available for differentiating between oxyhemoglobin and deoxyhemoglobin.

\begin{figure}[htcb]
\centergraphics{kuvat/hemoglobin_spectra.eps}
\caption{The absorbance spectra of different species of hemoglobins.\ \cite{Tobin1998}}
\label{fig:absorption_curves}
\end{figure}

The basis of pulse oximetry lies on photoplethysmography which is an optical method of measuring the volume change of a microvascular bed of tissue. A plethysmographic signal, shown in figure \ref{fig:pleth}, is usually obtained by transmitting light through an extremity such as a fingertip or an earlobe and measuring the throughput on the other side as explained before. The signal waveform is generated by the cyclic changes in the absorbance of the tissue, being mostly due to pulsatile arterial blood volume induced by cardiac activity. This waveform is often known as the \textit{AC} component of the plethysmographic signal and its amplitude is usually between 1-10\% of the total signal, illustrated in figure \ref{fig:absorbance_components}. The rest of the light is absorbed by or reflected by different tissues and non-pulsating venous blood, causing a large \textit{quasi-DC} offset in the signal. This offset is usually plainly referred to as the \textit{DC} component since even though factors such as respiratory activity, sympathetic nervous system activity and thermoregulation can induce some variance in it, the changes are much slower than the one induced by the heart.\ \cite{Allen2007}

\begin{figure}[htcb]
\centergraphics{kuvat/pleth.eps} %http://science.kingston.ac.uk/bpsrg/images/ECG.jpg
\caption{A photoplethysmographic signal shown with a simultaneous ECG signal. \cite{KingstonUniversity}}
\label{fig:pleth}
\end{figure}

\begin{figure}[htcb]
\centergraphics{kuvat/absorbance_components.eps}
\caption{A demonstrative illustration of absorbance components in photoplethysmographic measurement (not in scale): the AC amplitude of the signal is small compared to the DC level, usually in the range of 1-10\%.}
\label{fig:absorbance_components}
\end{figure}

The measurement principle of photoplethysmography is founded on the Beer-Lambert law that describes light attenuation through a sample of homogeneous non-scattering medium hosting an absorbent:

\begin{equation}
	I = I_0 e^{-\epsilon_\lambda \cdot c \cdot l} ,
\label{eq:Beer-Lambert}
\end{equation}

where $I$ is the intensity of transmitted light, $I_0$ is the initial light intensity, $\epsilon_\lambda$ the extinction coefficient of the absorbent at the wavelength $\lambda$, $c$ the concentration of the absorbent and $l$ the optical path length of light (in other words, the thickness of the medium). If more than one absorbent is present in the medium, the total light attenuation can be obtained as a linear superposition of attenuations of each absorbent. Transmittance is defined as the ratio of transmitted and the initial light intensity, and absorbance as the negative natural logarithm of transmittance, which can be expressed as:

\begin{eqnarray}
T &=& \frac{I}{I_0} = e^{-\epsilon_\lambda \cdot c \cdot l} \\
A &=& - ln T = \epsilon_\lambda \cdot c \cdot l 
\label{eq:transmittance}
\end{eqnarray}

Since in pulse oximetry we're only interested in the time variant part of the absorbance that relates to the pulsatile arterial blood, the time-dependent AC component of the absorbance can be separated from this model by differentiating equation (\ref{eq:transmittance}). In the case of monochromatic light, $I = I_\lambda$, the equation takes the following form:

\begin{equation}
  \frac{\partial A_\lambda}{\partial t} = -\frac{\partial I_\lambda}{\partial t \cdot I_\lambda} = \epsilon_\lambda \cdot c \cdot \frac{\partial l}{\partial t} .
  \label{eq:partial_absorbance}
\end{equation}

The total differential absorbance for a given wavelength $\lambda$ can now be expressed as follows:

\begin{equation}
  dA_{t,\lambda} = - \frac{\partial I_\lambda}{\partial t \cdot I_\lambda} \Delta t = \Delta l \sum_i{\epsilon_{i,\lambda} \cdot c_{i}} ,
  \label{eq:differential_absorbance}
\end{equation}

where the effects of all the different absorbents (hemoglobin types) having different absorption coefficients and concentrations are summed together. The differential can also be expressed in terms of the detected light intensity change:

\begin{equation}
  dA_{t,\lambda} = - \frac{\Delta I}{I} \cong \frac{I_{AC,\lambda}}{I_{DC,\lambda}} ,
  \label{eq:differential_absorbance_2}
\end{equation}

where $I_{AC,\lambda}$ is the time dependent and $I_{DC,\lambda}$ the time independent component of the measured transmission at wavelength $\lambda$. Pulse oximeters use this approximation to compare measured absorbances at red (\textit{R}) and infrared (\textit{IR}) wavelengths in order to estimate the ratio of oxyhemoglobin and reduced hemoglobin, or more often, the percentage of oxyhemoglobin in total hemoglobin content, denoted \textit{S\subscript{p}O\subscript{2}}. The relation of absorbances to oxygen saturation is shown in equation (\ref{eq:SpO2}).

\begin{eqnarray}
  \frac{dA_R}{dA_{IR}} &=& \frac{I_{AC,R} / I_{DC,R}}{I_{AC,IR} / I_{DC,IR}} \\
                       &=& \frac{( \epsilon_{O_{2}Hb,R} \cdot c_{O_2Hb} + \epsilon_{RHb,R} \cdot (1 - c_{O_2Hb} ) ) \Delta l}%
                       					{( \epsilon_{O_{2}Hb,IR} \cdot c_{O_2Hb} + \epsilon_{RHb,IR} \cdot (1 - c_{O_2Hb} ) ) \Delta l} \\
                       &=& \frac{S_{p}O_{2} \cdot \epsilon_{O_{2}Hb,R} +  (1 - S_{p}O_{2}) \cdot \epsilon_{RHb,R}}%
                      					{S_{p}O_{2} \cdot \epsilon_{O_{2}Hb,IR} +  (1 - S_{p}O_{2}) \cdot \epsilon_{RHb,IR}} \label{eq:SpO2}
\end{eqnarray}

The equation (\ref{eq:SpO2}) clearly shows that the simplified ideal model assumes that the only absorbents present in arterial blood are oxyhemoglobin and reduced hemoglobin. In most cases it poses no problem since the normal levels of carboxy- and methemoglobins are rather low \cite{Webster1997}, but some drugs and dyes in the circulation may have a big, unwanted effect on the measurement. The ideal theory can be generalized for \textit{n} wavelengths and \textit{m} hemoglobin types (\textit{n} being greater than or equal to \textit{m}) to improve either on the number of substances measured or the accuracy of measurement \cite{Aoyagi2007}, but it is not in the scope of this thesis.

Although the basis of oximetry is the Beer-Lambert law it can't be directly applied since the law only holds for monochromatic light through homogeneous and isotropic medium with no scattering and no association or dissociation of absorbing molecules. Since photoplethysmography of human tissue clearly doesn't fulfill these boundary conditions, empirical calibration is needed. \cite{Webster1997, Mendelson1992}

\subsection{\spo Measurement}

Arterial blood oxygen saturation measured by pulse oximetry (S\subscript{p}O\subscript{2}) is an estimate of the functional arterial oxygen saturation S\subscript{a}O\subscript{2}, which is the actual percentage of oxyhemoglobin of all the functional hemoglobin in the blood.

\begin{equation}
S_aO_{2(func)} = \frac{O_2Hb}{O_2Hb + RHb} \cdot 100\%
\label{SaO2}
\end{equation}

\begin{figure}[htcb]
\centergraphics{kuvat/measurement_setup.eps}
\caption{A typical S\subscript{p}O\subscript{2} measurement setup.}
\label{fig:SpO2_measurement}
\end{figure}

\begin{figure}[htcb]
\centergraphics{kuvat/measurement_cycle.eps}
\caption{A typical measurement cycle of a pulse oximeter.}
\label{fig:measurement_cycle}
\end{figure}

A typical pulse oximeter measurement setup is shown in figure \ref{fig:SpO2_measurement}. The sensor has the two LEDs on one side of a fingertip and a photodetector on the other side, and by pulsing the LEDs as indicated in figure \ref{fig:measurement_cycle} with a sufficient frequency, both of the photoplethysmographic signals can be obtained concurrently from the gathered data points. The duty cycle of the LEDs is usually between 5-10\% to allow for bigger peak currents without overheating the sensor -- as the sensor is used continuously in contact with skin, the temperature shouldn't rise much over normal body temperature \cite{ISO80601-2-612011}.

The measurements are processed with a suitable algorithm detecting and filtering out the most common disturbances explained in section \ref{section:limitations} and used to calculate the S\subscript{p}O\subscript{2} value by comparing their normalized differential absorptions given in equation (\ref{eq:differential_absorbance_2}). This produces a ratio \textit{R}, also known as a \textit{ratio of ratios}:

\begin{equation}
R = \frac{I_{AC,R} / I_{DC,R}}{I_{AC,IR} / I_{DC,IR}} .
\label{eq:ratio_of_ratios}
\end{equation}

Equation (\ref{eq:SpO2}) states that this ratio depends only on the blood oxygen saturation -- due to its relative nature it's unaffected by tissue thickness, skin pigmentation, blood volume, cardiac output, semiconductor nonlinearities or other variables that directly affect the absolute values of both of the pleth signals. The relationship between oxygen saturation and \textit{R} can be derived from the absorption spectra of the hemoglobin species to some extent using the ideal model, but for more accurate real-life measurements pulse oximeters use empirically gathered calibration data. The data is usually collected using a more accurate reference device, a co-oximeter, which requires a blood sample to be drawn from the subject.\ \cite{Webster1997}

Modern pulse oximeters calculate other parameters as well in addition to S\subscript{p}O\subscript{2} from the same measurement: heart rate, perfusion index and plethysmographic variability index with a visualization of the plethysmographic waveform provide an excellent tool for the doctor to estimate the patient's state.\ \cite{MasimoPVI, Allen1993}

\subsection{Limitations of Pulse Oximetry}\label{section:limitations}

It's to be noted that the S\subscript{p}O\subscript{2} figure is only relative and doesn't take the total amount of hemoglobin into account -- it doesn't tell the observer anything about the blood's total oxygen content. None the less, it's a good indicator for any possible disturbances in the respiratory and cardiovascular systems and in many cases serves as an early warning system for various critical conditions. Still, even though an ICU standard, pulse oximetry has some other drawbacks and limitations that have to be taken into account when assessing its reliability in clinical use and to avoid misinterpretation of measurements. They range from technical issues such as electromagnetic and physical interference to physiological limitations affecting the optical properties of the tissue, and some of them are discussed here.

\subsubsection{Ambient Light}

Probably the most obvious issue with oximetry is the ambient light that inevitably will find its way to the photodetector and is summed up in the measurement signal: if it contains some harmonic frequencies of the measurement cycle it may cause significant DC error and in any case add a lot of noise. Especially fluorescent and xenon lights in operating rooms have been known to cause both falsely normal and abnormally high readings even with the probe off patient. The effect of ambient light can be diminished physically by covering the probe with an opaque shell and by using measurement frequencies the harmonics of which don't coincide with the harmonics of the electric grid frequency.\ \cite{Ralston1991}

While the actions above are essential in limiting the ambient noise, a more sophisticated method of diminishing the effect exists. If the measurement cycle is short enough, by measuring the ambient level associated to each LED pulse (figure \ref{fig:ambient_sampling}) and subtracting the value from the result one can obtain a rather ambient-free result with only adding the noise of another sampling and conversion. The frequency response depends on the delay between the ambient and LED pulse measurements but in any case noise attenuation is of type 1/f. Considering that the bandwidth of interest is usually much lower than the sampling frequency and that of line frequency, this method blocks most of the noise on that band.

\begin{figure}[htcb]
\centergraphics{kuvat/ambient_sampling.eps}
\caption{A typical S\subscript{p}O\subscript{2} sampling sequence with ambient removal.}
\label{fig:ambient_sampling}
\end{figure}

\subsubsection{Patient Related Sources of Noise}

Another significant error source in pulse oximetry are motion artifacts. Given that the magnitude of the pulsatile signal is only percents of the total light transmission, any motion of the patient causing changes in local blood pressures and possibly sensor movement in respect to the tissue may greatly disturb the measurement. Caused by shivering, seizures or voluntary movements, the \spo algorithm may not detect the motion artifacts if their waveform resembles a normal plethysmographic signal which usually will lead to false readings and possibly an alarm. Modern oximeters utilize advanced signal processing methods to improve their accuracy even during motion artifacts.\ \cite{Webster1997, Lee2004}

A source for error that is especially bothersome in intensive care is the effect of low peripheral blood perfusion which can be caused by e.g.\ hypothermia, low cardiac output, loss of blood and symphathetic nervous activity, all of which can be a reason for intensive care in the first place. When the pulsatile signal intensity decreases it becomes exceedingly difficult for the measurement circuit to detect the plethysmographic signal from background noise. Pulse oximeters generally aren't able to operate if perfusion is very low, meaning an AC amplitude of 0.1\% of the total signal level or less, and produce an alarm instead.

Since most pulse oximeters are calibrated for normal, healthy people and only have two wavelengths, they have to assume that dyshemoglobin levels are sufficiently low. In some cases though, for example in carbon monoxide poisoning or methemoglobinemia, the patient's dyshemoglobin levels can rise dangerously high, greatly diminishing the oxygen carrying capacity of the blood. The spectral absorbance of carboxyhemoglobin greatly resembles the one of oxyhemoglobin, resulting in falsely high readings when carboxyhemoglobin concentration is elevated \cite{Barker1987}. Small amounts of methemoglobin tend to drive the oximeter reading falsely high, but with levels of over 20 percent the reading settles around 85\% \cite{Barker1989}. In addition, some drugs cause methemoglobinemia as a side effect which means that when administering them it should be noted that the pulse oximeter calibration isn't valid anymore \cite{Webster1997}.

Also dyes and drugs, or any substances, for that matter, in the blood that have an uneven absorption spectrum in the pulse oximetry range can affect the reading. The standard intravenous imaging dyes methylene blue, indocyanine green and indigo carmine have significant absorptions at the 660 nm wavelength but only minor absorptions at 940 nm \cite{Sidi1987, Scheller1986}. Thus, the use of these substances falsely decreases S\subscript{p}O\subscript{2} readings.

To summarize, it's extremely important for the physician to be aware of the various limitations of pulse oximetry and to apply another measurement technology when there's a known risk of false readings.\ \cite{Webster1997, Trivedi1997}


\subsection{Conventional System Design}

Pulse oximetry solutions usually resemble a configuration such as in figure \ref{fig:oximeter_block_diagram}: the system is controlled by a microprocessor which has inputs and outputs for all the peripherals and provides timing and operating levels for the analog measurement circuit, also known as the analog front end (\textit{AFE}). The AFE in turn provides the processor with measurements through the high-precision ADC for post-processing. The front end is usually implemented with a number of discrete components and integrated circuits, the blocks in the diagram roughly corresponding to the final setup. To be able to operate on a wide dynamic range the front end must have several processor-selectable LED current options and amplifier gain settings.

\begin{figure}[htcb]
\centergraphics{kuvat/oximeter_block_diagram.eps}
\caption{A typical block diagram of a pulse oximeter showing the Analog Front End, the Processor and some peripherals. \cite{focus.ti.com}}
\label{fig:oximeter_block_diagram}
\end{figure}

The analog front end can be divided further into two blocks: the LED driver and the receiver. The driver can be implemented as a \textit{common cathode} circuit where each LED has its dedicated anode wire and the cathodes are connected to a shared ground, or a \textit{back-to-back} circuit where the LEDs are connected to the same wires but with reversed polarities and are driven by an H-bridge -- this has the advantage of needing only two wires from the circuit board to the probe but also the disadvantage of the voltage drop of the added switch in the signal path. Both configurations are shown in figure \ref{fig:led_configurations}. The driver also includes a transconductance amplifier providing a controlled current proportional to its input voltage from the DAC, eliminating the effect of variable resistances in different probes and wires.

\begin{figure}[htcb]
\centergraphics{kuvat/led_configurations.eps}
\caption{Different LED configurations, using a different number of wires and a different driver.}
\label{fig:led_configurations}
\end{figure}

The basic receiver configuration, in addition to the photodetector, includes a transimpedance amplifier to convert the photodetector current into a voltage signal, a low-pass filter, a sample-and-hold circuit and a high-precision A/D converter as illustrated in figure \ref{fig:receiver}. Typically the ADC has a precision of at least 18 bits, often even more. To take full advantage of the ADC, the analog front end must be designed to have as good a signal-to-noise ratio (\textit{SNR}) as possible, though; usually this means that the operational amplifier component must be of high quality and with extremely stable reference voltages. \cite{Webster1997}

\begin{figure}[htcb]
\centergraphics{kuvat/receiver.eps}
\caption{A simplified example of a typical pulse oximeter receiver. The separation of AC and DC signal paths is optional and is recommended only in the case that the ADC has a low number of bits. \cite{Webster1997}}
\label{fig:receiver}
\end{figure}

Considering the low signal levels, noise is an important factor when designing the front end, and ensuring a signal-to-noise ratio (SNR) as high as possible is vital for the device's operation in more challenging situations. When designing a system like this there's bound to be some trade-offs between performance, power and price, some of which are discussed next.

In the measurement there's a strong relationship between LED current, pulse width and receiver gain. Ideally, the emitted light would be as bright and continuous as possible for the receiver to get the strongest possible signal, but in reality the LED drive power usage should be kept as low as possible for both battery life and probe temperature reasons. Therefore the multi-parameter optimization goal is to minimize LED power while maintaining such a signal-to-noise ratio that the measurement is reliable in all the designed use cases.

The noise produced by the LED driver doesn't account much to the overall system performance -- after all, the signal is strong while thermal noise is rather constant. The main sources for electrical noise in the system indeed are in the receiver side, mainly the photodiode itself, its cabling and the trans-impedance amplifier. Because of the as-low-as-possible emitted light intensity, the photodiode may exhibit noticeable shot noise in addition to thermal and 1/f noise present in all conductors. On top of that, shortening the LED pulse increases the bandwidth demand of the receiver. Since (white) noise energy is directly proportional to the bandwidth, this all leads to the fact that there's a big trade-off between LED power and receiver noise when controlling both the pulse intensity and the duty factor. \cite{Maas2005, Glaros2009}

Short pulse length can be an upside as well, though: the shorter the pulse, the more closely the actual measurement and the ambient measurement can be packed, better eliminating ambient noise. By using a higher pulse repetition frequency the measurements can be averaged and decimated down to the target sample frequency, decreasing total noise.