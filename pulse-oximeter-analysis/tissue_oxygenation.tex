\clearpage
\section{Tissue Oxygenation}

The metabolic reactions that fuel living cells need a continuous supply of oxygen (O\subscript{2}). These reactions oxidize nutrient molecules for energy and as a byproduct produce carbon dioxide (CO\subscript{2}) which has to be removed to prevent toxic acidity in cells. The exchange of these gases, O\subscript{2} and CO\subscript{2}, relies on the cooperation of the cardiovascular and respiratory systems, either of which failing would result in severe consequences: severe, irreversible cell damage can occur if oxygen supply is prohibited due to the fact that the tissues have only minimal oxygen reserves -- for example the brain will suffer from continued oxygen deprivation in a matter of minutes. Therefore it's essential to monitor oxygen delivery during anesthesia, critical care and other high-risk situations.\ \cite{Tortora2000}

\subsection{The Cardiovascular System}

The primary function of the cardiovascular system is to deliver oxygen, nutrients, signal molecules and other substances to all tissues and cells throughout the body and to remove the waste produced by them, such as carbon dioxide (CO\subscript{2}). In short, the cardiovascular system is a network of blood vessels that provides a means of transportation for these substances between the tissues and functional organs, powered by the heart.\ \cite{Tortora2000}

\subsubsection{Structure and Operation of the Cardiovascular System}

The mammal cardiovascular system is composed of two circuits, the pulmonary and the systemic one, and the heart that simultaneously pumps blood into them both (see figure \ref{fig:cardiovascular_system}). The pulmonary loop goes from the right ventricle of the heart to the lungs where blood is oxygenated, carbon dioxide is extracted and the blood directed back to the left atrium. Fresh, oxygen rich blood is then pumped by the left ventricle into systemic circulation through arteries and finally into capillaries where the exchange of blood-carried substances takes place. The now deoxygenated blood is carried back to the heart by veins and the loop continues.

\begin{figure}[htcb]
\centergraphics{kuvat/cardiovascular_system.eps} % http://www.urgomedical.com/var/ezflow_site/storage/images/media/images/venous-system-02/2059-1-eng-GB/venous-system-02.jpg
% http://www.urgomedical.com/Pathophysiologies/Compression/The-venous-system/In-the-cardiovascular-system
\caption{The structure of the cardiovascular system. \cite{urgomedical.com}}
\label{fig:cardiovascular_system}
\end{figure}

The cardiovascular system also includes a blood flow regulation system that utilizes several interconnected negative feedback systems that work both together and in parallel to adjust the amount and the distribution of blood flow within the body. The main function of this system is to ensure that all tissues receive adequate oxygenation while optimizing overall thermoregulation and energy consumption.

The control center for the cardiovascular system is the cardiovascular center in the brain stem. It receives input signals from various sensory receptors all over the system monitoring blood pressure, blood volume, blood acidity (pH), body temperature and body movements. Based on this input, the cardiovascular center adjusts the cardiac output (heart rate and stroke volume) and the resistance (diameter) of blood vessels to maintain an adequate blood pressure and tissue oxygenation in all parts of the system. Cardiac output, being defined as the product of heart rate and stroke volume, is essentially the total flow of blood through the heart and thus is associated with the total oxygen demand of the body. Blood vessel resistance, on the other hand, is a localized quantity and the resistance distribution between organs or body parts rarely is even. The resistance is controlled by the cardiovascular center in the brain using smooth muscles lining the walls of blood vessels either to constrict or to dilate them.\ \cite{Tortora2000}

\subsubsection{Blood and Hemoglobin}

Blood has two primary components. The liquid part with different proteins and other substances dissolved in it is called \textit{plasma}, and it usually constitutes about 55\% of total blood volume. The other 45\% are blood cells and cell fragments out of which normally over 99\% are red blood cells (\textit{RBCs}). The rest are white blood cells and platelets. Therefore the percentage of total blood volume occupied by RBCs is normally in the range of 38-54\% and is called the \textit{hematocrit}.\ \cite{Tortora2000}

Oxygen, in its basic molecular form, doesn't dissolve easily in plasma. That is the reason red blood cells, also known as \textit{erythrocytes} and seen in figure \ref{fig:rbc}, have the sole purpose of oxygen transportation and are highly specialized for it. Their bi-concave shape and the lack of most intracellular organs give them excellent surface-to-volume ratio for gas exchange by diffusion and frees the intracellular space for hemoglobin. Due to this, each microscopical erythrocyte is able to contain about 280 million hemoglobin molecules - roughly a third of the cell's mass.\ \cite{Tortora2000}

\begin{figure}[tcb]
\centergraphics{kuvat/rbc.eps}% http://mycozynook.com/22_11HemoglobinO2-L.jpg, http://images.medicinenet.com/images/illustrations/blood_cells.jpg
\caption{Hemoglobin molecule and the red blood cell.\ \cite{mycozynook.com, medicinenet.com}}
\label{fig:rbc}
\end{figure}

A hemoglobin molecule is a protein complex with four polypeptide chains and four non-protein pigments called \textit{hemes}. Each heme is associated with a polypeptide chain and contains a Fe\superscript{2+} ion, able to reversibly bind an oxygen molecule; oxygen is picked up in the lungs and the reaction is reversed in the tissues, releasing the oxygen molecule that diffuses first into the interstitial fluid and then into cells. The polypeptide chains of hemoglobin also take care of transferring about 23\% of the carbon dioxide produced in cells, the rest being dissolved into plasma.\ \cite{Tortora2000}

\subsubsection{Dysfunctional Hemoglobins}

The vast majority of hemoglobin in circulation is either fully oxygenated hemoglobin (\textit{oxyhemoglobin}, O\subscript{2}Hb) or reduced hemoglobin that isn't oxygen-saturated (also called \textit{deoxyhemoglobin}). These two hemoglobin species are referred to as \textit{functional hemoglobins} because they can execute the oxygen transportation task. In addition to these, there exist several other species of hemoglobin in the blood that have lost their oxygen carrying capability and are thus called \textit{dysfunctional hemoglobins} (\textit{dyshemoglobins}). Normally dyshemoglobins constitute less than 3\% of total hemoglobin content, a level that doesn't really affect oxygenation. There are some conditions, though, where the concentration of dyshemoglobins may rise significantly, reducing the amount of functional hemoglobin available and so endangering tissue oxygenation.

Dyshemoglobins are modified hemoglobins that form when functional hemoglobin reacts with other substances, including carbon monoxide and hydrogen sulfide. The most common dyshemoglobin, carboxyhemoglobin (COHb), is formed when the process that normally binds oxygen to hemoglobin captures a carbon monoxide molecule instead. Carbon monoxide's binding affinity in this process is ca. 210-fold compared to oxygen's, rendering the hemoglobin molecule essentially inert for oxygen. Another common dyshemoglobin is methemoglobin (MetHb) that is the result of an oxidation reaction of the hemoglobin molecule: when the iron ion of the heme is oxidized from Fe\superscript{2+} to Fe\superscript{3+}, the heme loses its ability to bind oxygen and is rendered useless for its oxygenation purpose.\ \cite{Webster1997}

\subsection{The Respiratory System}

In biology, respiration is defined as the sum of chemical and physical processes in an organism by which oxygen is conveyed to tissues and the oxidation products are given off to the parent medium (in the case of mammals, air). It can be divided into three elementary steps, which are pulmonary ventilation, external respiration and internal respiration. \textit{The respiratory system} is responsible for the first two (air-blood interface), the main objective being the gas exchange between air and blood. The latter one, internal respiration, refers to the exchange of gases in systemic capillaries, being mostly a responsibility of the cardiovascular system.\ \cite{Tortora2000}

\subsubsection{Anatomy of the Respiratory System}

The respiratory system consists of the nose, pharynx (throat), larynx (voice box), trachea (windpipe), bronchi and lungs (figure \ref{fig:respiratory_system}). Its main function is to get fresh air into contact with as much diffusion surface as possible which is achieved by the \textit{conducting portion} of the respiratory system directing air to lungs, which in turn have a huge number of extremely small air conduits that maximize the air's surface-to-volume ratio. This is essential for effective gas exchange through the respiratory membrane since the amount of diffused gases is directly proportional to the diffusion surface, not total volume. It's figurative that the average total lung volume in an adult is between 5 and 6 liters while the total surface area is roughly 70 m\superscript{2}. This is the equivalent of a standard bucket and a tennis court.\ \cite{Tortora2000}

\begin{figure}[b]
\centergraphics{kuvat/respiratory_system.eps}% Wikimedia commons; http://commons.wikimedia.org/wiki/File:Illu_bronchi_lungs.jpg
\caption{The human respiratory system.\ \cite{commons.wikimedia.org}}
\label{fig:respiratory_system}
\end{figure}

%\subsubsection{Control of Respiration}

Pulmonary ventilation, commonly known as breathing, is driven by the contraction and relaxation of respiratory muscles. The muscles, mainly the diaphragm, increase the volume of the thoracic cavity when contracting, causing a pressure difference between the lungs and the atmosphere which forces air to flow into the lungs. When said muscles relax the thoracic cavity returns to its normal volume and air is forced out.

The diaphragm and other respiratory muscles are skeletal muscles, which means they must be controlled directly with nerve impulses. Most of the time this control is involuntary but can be overridden by voluntary actions. The respiratory center is located in the brain stem and is functionally divided into three areas: the \textit{medullary rhythmicity area} provides the basic rhythm of respiration but it can be modified, along with the volume of inspiration, by the \textit{pneumotaxic} and \textit{apneustic areas} according to inputs from other parts of the brain and sensory receptors. Like blood circulation, also respiration is regulated in response to various signals originating from all around the body and measuring quantities such as oxygen and carbon dioxide levels and blood pressure. In fact, some of these receptors are multifunctional, being used by both the respiratory and the cardiovascular centers.\ \cite{Tortora2000}

\subsubsection{Gas Exchange}

The transfer of oxygen and carbon dioxide in the air-blood and the blood-tissue interfaces is based on the differences in partial pressures of these gases throughout the cardiovascular and respiratory systems. Therefore the physics of partial pressures is worth looking into.

In gas mixtures, such as air, each component gas has a partial pressure that is independent of the other gases and is denoted as \textit{p} - it's the pressure the gas would have if it alone occupied the volume. Therefore the sum of all the partial pressures equals the total pressure of the gas mixture. Diffusion can be modelled easily with partial pressures: when there's a partial pressure gradient for a gas, there will be a net flow of that gas in the direction of the gradient until equilibrium is reached.%\ \cite{Fysiikankirja}

The exchange of gases in alveoli, as well as in systemic capillaries, is driven by diffusion. Since the partial pressure of oxygen (p\subscript{O\subscript{2}}) is higher in the air in alveoli (ca. 140 hPa at sea level) than in the blood entering the pulmonary capillaries (ca. 50 hPa) \cite{Thomas2004}, oxygen diffuses through the respiratory membrane into the blood. The p\subscript{O\subscript{2}} in the interstitial fluid in tissues is even lower than in the blood, causing oxygen molecules to diffuse through the capillary wall. Carbon dioxide behaves similarly but in the opposite direction due to its higher concentration in tissues compared to air.

In addition to diffusion, partial pressure is also the most important factor in determining how oxygen combines with hemoglobin and therefore tremendously affects the efficiency of oxygen transportation. This is due to the fact that over 98\% of transported oxygen is bound to hemoglobin since oxygen doesn't dissolve easily in water. The binding affinity is dependent on the p\subscript{O\subscript{2}} of blood according to the oxygen-hemoglobin dissociation curve in figure \ref{fig:oxygen_dissociation}, which shows the oxygen saturation (percentage of oxyhemoglobin out of total functional hemoglobin) as a function of p\subscript{O\subscript{2}}. In the alveolar p\subscript{O\subscript{2}}, 140 hPa (105 mmHg), oxygen binds very well with hemoglobin, resulting in a saturation level of 97\% of arterial blood. When the blood reaches the systemic capillaries the diffusion starts, lowering the blood's p\subscript{O\subscript{2}} and inducing oxygen dissociation from hemoglobin. After the gas exchange the p\subscript{O\subscript{2}} of venous blood is about 50 hPa (40 mmHg, equal to the level in tissues), meaning that hemoglobin is still roughly 75\% saturated. In conclusion, only 25\% of the blood's total oxygen content is released in tissues. \cite{Tortora2000}

\begin{figure}[htcb]
\centergraphics{kuvat/hb_saturation_curve.eps} % http://www.molecularstation.com/molecular-biology-images/data//505/Hb-saturation-curve.png
\caption{Oxygen dissociation curve of hemoglobin. \cite{molecularstation.com}}
\label{fig:oxygen_dissociation}
\end{figure}

\subsection{Hypoxia}

Hypoxia, by definition, is a condition where tissue lacks adequate oxygen supply. It can be a local or a general condition depending on the circumstances. In all cases, though, disturbing the oxygen delivery to cells greatly disrupts cell metabolism, eventually leading to cell necrosis and tissue damage. This may occur in just a few minutes, first affecting major oxygen consumers such as the brain, the heart and other vital organs. It often happens that the first symptoms of hypoxia, mainly skin discoloration in the case of unconscious patients, go unnoticed because as subtle changes they are difficult to recognize and measure; when the condition is finally diagnosed after worsened symptoms it may already be too late.

The are various possible underlying reasons for hypoxia, related to different parts of the oxygen delivery system. They include, but are not limited to, for example low p\subscript{O\subscript{2}} in arterial blood due to low p\subscript{O\subscript{2}} in air in high altitudes or closed spaces (hypoxic hypoxia), low amount of functional hemoglobin in blood due to poisoning or anemic conditions (anemic hypoxia), poor or blocked blood flow to tissues (stagnant hypoxia) or tissues being unable to utilize oxygen (histotoxic hypoxia). The reason for hypoxia could even be as simple as the patient not breathing.\ \cite{Tortora2000, Webster1997}

\subsection{Anesthesia}

When a patient is kept in anesthesia e.g.\ for a surgical procedure it's common that the patient has been intubated, having a respirator taking care of the patient's breathing. In these conditions it is extremely important to monitor the patient's \spo since otherwise the machine can't compensate for increased or decreased oxygen consumption as a healthy person would by breathing harder. Also leaks and other malfunctions that severely affect the patient's respiratory function can happen.\ \cite{Pierce1998}

Another factor is the importance of pulse oximetry in the recovery room. It often happens that a patient recovering from anesthesia has received a slight overdose of the anesthetic agent or for other reasons exhibits reduced respiratory function. Detecting such conditions greatly improves patient outcome, saving them from unnecessary complications, and therefore pulse oximetry monitoring is recommended as a standard procedure also for recovery.\ \cite{Morris1987}