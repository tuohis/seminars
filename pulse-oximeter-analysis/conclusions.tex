\clearpage
\section{Conclusions}

The work done in this thesis was a part of a project which develops a new generation of pulse oximetry measurement hardware. The development model of the embedded software used with the hardware was fine-tuned to serve both the evaluation process and product development, and indeed most of the software developed for the purpose of performance evaluation will also be used in the final product. The performance analysis results themselves were essential in the hardware evaluation process -- in fact, they clearly pointed out a flaw in the design that otherwise probably would have gone unnoticed. The analysis had to be done using a black box approach due to the hardware being tightly packed and shielded, rendering measurements of the individual subsystems impossible to do; that is why the output of the system's ADC was the only data source used.

\subsection{Software}

A completely new piece of embedded software was written for the new measurement circuit's microcontroller using Embedded C++ and object-oriented design. The software has total control of the hardware, taking care of the timing and signal levels of the pulse oximetry measurement and handling communication between the measurement circuit and the host. Hardware timers, DMA and interrupts were used. The development of the software and especially the priority order of certain components was highly guided by the evaluation process, yet everything was done keeping in mind that the result would be a usable and maintainable final product. The goal was reached very well.

In addition to the embedded software, also a host program with a graphical user interface was written using Matlab. It served as a signal acquisition and data analysis tool during the verification process, providing a two-way interface to the measurement circuit. A lot of the tool's code can most probably be re-used in further evaluations as only the hardware setting editor is system-dependent.

\subsection{Performance Analysis}

Noise requirements for the hardware were calculated based on the signal processing algorithm used and the performance specifications of the \spo measurement. The results were that a signal-to-noise ratio of ca.\ 98 dB would be required to reliably measure \spo from a low perfusion pleth signal with IR modulation of 0.02\%. This was also tested empirically and it was concluded that a signal-to-noise ratio of approximately 100 dB was needed when using a simulator to generate the pleth waveform; the setting also represents a bad-case scenario as the simulator's spectral noise distribution has significant spikes at the fundamental cardiac frequency and its harmonics.

The performance of the hardware was assessed by systematically measuring the amount of noise present in the output signal with different system settings. By fitting the results to the model of the system it was possible to pinpoint the major sources of noise, i.e.\ the parts of the system that weren't performing optimally. The measurements were made as sets, modifying only one operating parameter in each. The most important parameters were receiver gain, sampling time, LED current and sensor type.

The performance tests showed that the receiver performed well with a DC signal. Once a pulsed signal which is used in normal operation was introduced the performance dropped significantly. The tests ruled out the transmitter as the noise source, indicating that a component in the receiver chain was flawed. In addition to measuring the noise present in the output signal, the correlation of different channels was examined as well to prove that the observed added noise component in the signal channel didn't correlate with its corresponding ambient channel and therefore was most likely thermal noise affecting only one of the signals or noise the magnitude of which depended on the signal level. Further on, the tests allowed to isolate the problem in the sampling and conversion units. Finally the bug was found in the buffer feeding the ADC and a fix was made by the hardware team, increasing system performance significantly.

In the end it was determined that the new measurement circuit will fulfill the design specifications set for it after the bugs found are fixed. The circuit will be used in a range of GE's upcoming products.