\section{Johtop��t�kset}

Arpikudoksen syntyminen on merkitt�v� ongelma kroonisesti implantoitujen aivoelektrodien yhteydess�, mutta viime vuosina on kehitelty useita menetelmi� useilla eri l�hestymistavoilla arpeutumisen v�hent�miseksi. Lupaavia tuloksia on saatu sek� uudenmallisten elektrodien, elektrodimateriaalien ett� erilaisten elektrodipinnoitteiden tiimoilta. Todenn�k�isesti paras tulos syntyy yhdistelem�ll� n�it� tekniikoita hallitusti ja kenties tulevaisuudessa ottamalla mikrofluidistiikka avuksi l��keaineiden kohdennetussa annostelussa. My�s implantin johdotukseen tulee kiinnitt�� huomiota, tavoitteena minimoida implantin mikroliike aivokudokseen n�hden.

Yleisesti ottaen ala kehittyy nopeasti luotettavan aivo-tietokone-k�ytt�liittym�n tarjoamien ��rett�mien mahdollisuuksien, kuten ajatuksella ohjattavien keinoraajojen ansiosta ja mit� suurimmalla varmuudella jatkaa kehityst��n tulevina vuosina. Yhten� yksitt�isen� kehityskohteena tieteellisell� yhteis�ll� voisi olla yhdenmukaisten testiolosuhteiden luominen, jotta eri tutkimuksia voitaisiin luotettavasti vertailla.